For the system described by: 
\begin{equation}
\dot x = 
\begin{bmatrix}
  0 & 1 & 0\\
  0 & 0 & 1\\
  -1 & 4 & 5\\
\end{bmatrix}x + 
\begin{bmatrix}
  0\\
  0\\
  2\\
\end{bmatrix}u
\end{equation}
\begin{equation}
\dot x = 
\begin{bmatrix}
  0 & 1 & 0\\
  0 & 0 & 1\\
\end{bmatrix}x + 
\begin{bmatrix}
  0\\
  0\\
\end{bmatrix}u
\end{equation}
The following variables were chosen for the observer:

\begin{tabular}{r|l}
$L_0$ & $
\begin{bmatrix}
  0. & 1.\\
  0.5 & 0.\\
  0.1 & 0.25\\
\end{bmatrix}$\\
$F$ & $
\begin{bmatrix}
  -5. & 5. & 0.\\
  -5. & -5. & 0.\\
  0. & 0. & -5.\\
\end{bmatrix}$\\
\end{tabular}

These were used to calculate: 

\begin{tabular}{r|l}
$T$ & $
\begin{bmatrix}
  0.00264941 & 0.04032866 & 0.07280946\\
  -0.01191248 & 0.09910611 & -0.04631534\\
  0.00502183 & -0.0010917 & 0.02510917\\
\end{bmatrix}$\\
$L$ & $
\begin{bmatrix}
  55.62246651 & -6.38612161\\
  8.56733081 & 4.57523188\\
  -6.76939196 & 11.43266919\\
\end{bmatrix}$\\
\end{tabular}

The following is the outputs of the LQR system assuming the inputs are 0 and an initial estimate of state of $
\begin{bmatrix}
  1.\\
  -1.\\
  1.\\
\end{bmatrix}$

\image{two_two_output.png}{LQR system}{fig:two_two}