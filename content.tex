\question{1}{30pts}{
  A High performance helicopter has a model shown in \autoref{fig:helicopter}. The goal is to control the pitch
  angle $\theta$ of the helicopter by adjusting the rotor thrust $\delta$. The equations of motion of the
  helicopter are
  \image{resources/helicopter.png}{\phantom{x}}{fig:helicopter}
  \begin{equation}
    \frac {d^2\theta}{dt^2} = -\sigma_1 \frac{d\theta}{dt} - \alpha_1 \frac{dx}{dt} +n\delta
  \end{equation}
  \begin{equation}
    \frac{d^2x}{dt^2} = g\theta - \alpha_2 \frac{d\theta}{dt} - \sigma_2 \frac{dx}{dt} + g\delta
  \end{equation}
  Where x is the translation in the horizontal direction. For a miltary high-preformance helicopter we find:
  $\sigma_1 = 0.415$, $\sigma_2 = 0.0198$, $\alpha_1 = 0.0111$,$\alpha_2 = 1.43$, $n=6.27$, $g=9.8$ all in
  appropriate SI units.
  Find:
  \begin{enumerate}[(\alph*)]
  \item A state variable representation of this system
  \item The transfer function representation for $\frac{\theta(s)}{\delta(s)}$
  \item Use state variable feedback to achieve adaquate performances for the controlled system.
    Desired specifications include:\label{step:specifications}
    \begin{enumerate}[(\arabic*)]
    \item A steady-state for an input step command for $\theta_d(s)$, the desired pitch angle, less than
      20\% of the input step magnitude
    \item An overshoot for a step input command is less than 20\%
    \item a settling (with a 2\% criterion) time for a step command of less than 1.5 seconds
    \end{enumerate}
  \item If the state variable is not available, design the observer and control law to meet the design
    specifications included in part
  \end{enumerate}
}{
  \begin{enumerate}[(\alph*)]
  \item \subimport{./scripts/results/}{one_a.tex}
  \item \subimport{./scripts/results/}{one_b.tex}
  \item part c
  \item If the state variable is not available, design the observer and control law to meet the design
    specifications included in part
  \end{enumerate}
}{}
\question{2}{30pts}{
  The open loop system
  \begin{equation}
    \dot x =
    \begin{bmatrix}
      0 & 1 & 0 \\
      0 & 0 & 1 \\
      -1 & 4 & -3
    \end{bmatrix}x +
    \begin{bmatrix}
      0\\
      0\\
      2
    \end{bmatrix}u
  \end{equation}
  \begin{equation}
    y =
    \begin{bmatrix}
      0 & 1 & 0 \\
      0 & 0 & 1
    \end{bmatrix}x
  \end{equation}
  \begin{equation}
    x(0) =
    \begin{bmatrix}
      1 \\
      -1\\
      1
    \end{bmatrix}
  \end{equation}
  \begin{enumerate}[\arabic*)]
  \item Assume that x is availbale for state feedback. Design and LQR control law by letting $R=1$ and choosing
    $Q$ so that all the elements of the feedback gain $K$ have absolute value less than 50.\\
    Requirement: $\vert y_1(t)\vert \le 0.05$,$\vert y_2(t)\vert \le 0.05$, for all $t > 5$. Plot $y_1(t)$ and
    $y_2(t)$ in the same figure for $t \in [0,20]$ \label{step:2-1}
  \item Assume that only the output $y$ is available. Design an observer so that the poles of the observer are
    $-5 \pm j5, -10$. Choose the observer gain so that all the elments have absolute value less than 80. Form a
    closed loop system along with the LQR controller in step \ref{step:2-1}. Plot $y_1(t)$ and $y_2(t)$ in the
    same figure for $t \in [0,20]$
  \end{enumerate}
}{
  \begin{enumerate}[\arabic*)]
  \item Assume that x is availbale for state feedback. Design and LQR control law by letting $R=1$ and choosing
    $Q$ so that all the elements of the feedback gain $K$ have absolute value less than 50.\\
    Requirement: $\vert y_1(t)\vert \le 0.05$,$\vert y_2(t)\vert \le 0.05$, for all $t > 5$. Plot $y_1(t)$ and
    $y_2(t)$ in the same figure for $t \in [0,20]$ \\
    \subimport{./scripts/results/}{two_one.tex}
  \item Assume that only the output $y$ is available. Design an observer so that the poles of the observer are
    $-5 \pm j5, -10$. Choose the observer gain so that all the elments have absolute value less than 80. Form a
    closed loop system along with the LQR controller in step \ref{step:2-1}. Plot $y_1(t)$ and $y_2(t)$ in the
    same figure for $t \in [0,20]$
    \subimport{./scripts/results/}{two_two.tex}
  \end{enumerate}


}{}

\question{3}{40pts}{
  A cart with an inverted pendulum as seen in \autoref{fig:cart}
  \image{resources/inverted_pendulum.png}{\phantom{x}}{fig:cart}
  
  \begin{tabular}{r|l}
    n & control input(Newtons)\\
    y & displacement of the cart(meters)\\
    $\theta$ & angle of the pendulum(radians)
  \end{tabular}

  \begin{equation}
    x = \begin{bmatrix}
      y \\
      \dot y \\
      \theta \\
      \dot \theta
    \end{bmatrix}
  \end{equation}
  The control problems are
  \begin{enumerate}[\arabic*:]
  \item Stabalization: Design a feedback law $u$ $Fx$ such that $x(t) > 0$ for $x(0)$ close to the diagram
  \item For $x(0) = (0, 0, -\pi, 0)$, apply an impulse force $u(t) = u_{max}$ for
    $t \in [0,0.1]$ to bring 0 to a certain range and then switch to the linear controller so that $x(t) \to 0$.
  \end{enumerate}
  Assume that there is no friction or damping. The nonlinear model is as follows.
  \begin{equation}
    \begin{bmatrix}
      M+m & ml\cos(\theta) \\
      \cos(\theta) & l
    \end{bmatrix}
    \begin{bmatrix}
      \ddot y\\
      \ddot \theta
    \end{bmatrix} -
    \begin{bmatrix}
      u + ml\dot \theta^2 \sin \theta \\
      g \sin \theta
    \end{bmatrix}
  \end{equation}
  with
  
  \begin{tabular}{r|l}
    $m=1kg$ & mass of the pendulum\\
    $l=0.2m$ & length of the pendulum\\
    $M = 5kG$ & mass of the cart\\
    $g = 9.8\frac m {s^2}$ & mass of the cart
  \end{tabular}
  
  Linearize the system at $x=0$
  \begin{equation}
    \begin{bmatrix}
      M+m & ml \\
      1 & l
    \end{bmatrix}
    \begin{bmatrix}
      \ddot y\\
      \ddot \theta
    \end{bmatrix} -
    \begin{bmatrix}
      u \\
      g\theta
    \end{bmatrix}
  \end{equation}
  the state space description for the linearlized system.
  \begin{equation}
    \dot x = Ax + Bu
  \end{equation}

  Problems:
  \begin{enumerate}[\arabic*.]
  \item Find matrices $A$, $B$ for the state space equation.
  \item Design a feedback law $u-F_1x$  so that $A+BF_1$ has eignevalues as $-3\pm j3, -6, -8$. Build a simulink
    model for the closed loop linear system. Plot the response under initial condition
    $x(0) = (-1.5, 0, 1, 3)$.
  \item Build a simulink model for the original nonlinear system, verify that stabilization is achieved by
    $u=F_1x$ when $x(0)$ is close to the origin. Find the maximal $\theta_0$ so that nonlinear system can be
    stabalized from $x_0 = (0, 0, \theta_0,0)$
  \item For $x(0)=(0,0, \frac{\pi} 5,0)$, compare the response $y(t)$ and
    $\theta(t)$ for the linearized system and the nonlinear system under the same feedback $u - F_1x$
  \end{enumerate}
}{% work for question 3
  \begin{enumerate}[\arabic*.]
  \item \subimport{./scripts/results/}{three_one.tex}
  \item \subimport{./scripts/results/}{three_two.tex}
  \item \subimport{./scripts/results/}{three_three.tex}
  \item For $x(0)=\begin{bmatrix}0\\0\\\frac{\pi} 5\\0\end{bmatrix}$, compare the response $y(t)$ and
    $\theta(t)$ for the linearized system and the nonlinear system under the same feedback $u - F_1x$
  \end{enumerate}
}{}
